\documentclass[unicode, aspectratio=169]{beamer}
\usepackage[ipaex]{luatexja-preset}
\usepackage{luatexja}

\usepackage{lambdalf}
\usepackage{mydefault}
\usepackage{mymath}
\usepackage{myproof}
\usepackage{pgfpages}

\setbeameroption{show notes on second screen}

% === set theme === %
% \usetheme[compress]{Frankfurt}
% \usetheme[compress]{Execushares}
\usetheme[compress]{Singapore}
\usecolortheme{orchid}

% === magic === %
\AtBeginSection{
    \frame{\insertsectionhead}
}
\makeatletter 
\def\beamer@framenotesbegin{% at beginning of slide
     \usebeamercolor[fg]{normal text}
      \gdef\beamer@noteitems{}% 
      \gdef\beamer@notes{}% 
}
\makeatother
% ============= %

\addbibresource{mybib.bib}

\title{Staged Dependent Lambda Calculus with Equality Types}
\subtitle{Reasearch Seminar}
\author{Shuntaro Katsuda}
\date{\today}
\institute{Undergraduate School of Informatics, Kyoto University}
\titlegraphic{\includegraphics[scale=0.04]{./kuis_logo.jpeg}}

\begin{document}
    \frame{\titlepage}

    \section{Introduction to the system}

    \begin{frame}{Terms}
        \lamlfcirc \mysp adds staged computation of Davies's \lamcirc \cite{Davies1996} to  \lamlf \cite{pierce_2005} 
        \begin{columns}
            \begin{column}{.4\textwidth}
                \begin{equation*}\tag{terms}
                    \begin{aligned}
                        t \Coloneqq\\
                        &x\\  
                        &\lambda x: T. t\\
                        &t\mysp t \\
                        &\mynext{t}\\
                        &\myprev{t}\\
                        &\myid{t}\\
                        &\myidpeel{t}{x}{t}\\
                    \end{aligned}
                \end{equation*}
            \end{column}
            
            \begin{column}{.4\textwidth}
                \begin{equation*}\tag{types}
                    \begin{aligned}
                        T \Coloneqq\\
                        &X\\ 
                        &\textbf{Eq}_T\\
                        &\Pi x: T. T\\
                        &T\mysp t\\
                        &\bigcirc T\\
                    \end{aligned}
                \end{equation*}
            \end{column}
        \end{columns}
        \note{
            hogehoge
        }
    \end{frame}

    \section{Equality of terms, types and kinds}

    \begin{frame}{Equality of terms concerning stages}

        Do not allow equivalence of terms via code embedding inside \textrm{next}.

        \infrule[Q-$\bigcirc$\fauxsc{Beta}]{
            \Gamma \myvdash{n} t : T
            \andalso \Gamma \myvdash{n} T :: *
            \andalso m \le 1
        }{
            \Gamma \myvdash{n} \myprev{\mynext{t}} \myequiv{m} t : T
        }
        
        \begin{alertblock}{Disallow}
            $\Gamma \myvdash{n} \mynext{\mynext{3 + \myprev{\mynext{5}}}} \myequiv{0} \mynext{\mynext{3 + 5}}$
        \end{alertblock}

        \begin{exampleblock}{Allow}
            $\Gamma \myvdash{n} \mynext{3 + \myprev{\mynext{5}}} \myequiv{0} \mynext{3 + 5}$
        \end{exampleblock}

    \end{frame}

    \begin{frame}{Example of use}
        The $\textrm{\bigcirc}$\fauxsc{Beta} rule works well in disallowing code embedding inside \textrm{next}.
        \begin{alertblock}{Disallow}
            $\Gamma \myvdash{n} \mynext{\mynext{3 + \myprev{\mynext{5}}}} \myequiv{0} \mynext{\mynext{3 + 5}}$
        \end{alertblock}

        \begin{prooftree}
            \AxiomC{$\Gamma \myvdash{n} TODO$}
        \end{prooftree}
    \end{frame}

    \begin{frame}{Equivalence of types concerning stages}
        Because of dependent types, some types include terms.
        Stages need to be considered when comparing types.

        \begin{exampleblock}{Example}
            \[\Gamma \myvdash{n} \VEC 3 \equiv^{?}_{0} \VEC (1 + 2)\]
        \end{exampleblock}
        \note{
            前回は, type と kind について equivalence の index がありませんでしたが,
            よく考えたら必要なことが判明しました.
            これをどう定義したかについてお話します.
            とりあえずこの例では同じにしたいです.
         }
    \end{frame}

    \begin{frame}{Trickier Example}

        \begin{exampleblock}{Example}
            \[\Gamma \myvdash{n} \mynext{\VEC 3} \equiv^{?}_{0} \VEC (1 + 2)\]
        \end{exampleblock}

        \begin{block}{policy}
            Compare terms inside types at the current equivalence depth.
        \end{block}

        \note{
            Vector 3 が next の中に入っています.
        }
    \end{frame}
    
    \section{Evaluation of terms}
 
    \begin{frame}{Defining values}
        
        \begin{columns}
            \begin{column}{.4\textwidth}
                \begin{equation*}\tag{values}
                    \begin{aligned}
                        v \Coloneqq\\
                            &\mylam{x}{T}{t}\\
                            &\mynext{v}\\
                            &\myid{v}\\
                            &\myidpeel{v}{x}{v}\\
                        \end{aligned}
                \end{equation*}
            \end{column}
            \begin{column}{.4\textwidth}
                \begin{equation*}\tag{indexed terms}
                    \begin{aligned}
                        t' \Coloneqq\\
                        & \myval{t}{d} \mysp (d \in \mathbb{N}) \\
                    \end{aligned}
                \end{equation*}  

                \begin{equation*}\tag{indexed values}
                    \begin{aligned}
                        v' \Coloneqq\\
                            & \myval{v}{d} \mysp (d \in \mathbb{N})
                    \end{aligned}
                \end{equation*}
            \end{column}
        \end{columns}
    \end{frame}

    \begin{frame}{Meaning of the index}
        Index of values indicate in which stage we are trying to evaluate a term.
        \infrule[E-Next]{
            \myval{t}{d+1} \rightarrow \myval{t'}{d+1}
            \andalso d \le 1    
        }{
            \myval{\mynext{t}}{d} \rightarrow \myval{\mynext{t'}}{d}
        }
        
        \infrule[E-Prev]{
            \myval{t}{d+1} \rightarrow \myval{t'}{d+1}
            \andalso d \le 1    
        }{
            \myval{\myprev{t}}{d} \rightarrow \myval{\myprev{t'}}{d}
        }
    \end{frame}

    \begin{frame}{Rules where index is important}
        \infrule[E-StageBeta]{
            \myval{t}{d} \rightarrow \myval{t'}{d}
            \andalso d \le 1
        }{
                \myval{\myprev{\mynext{t}}}{d} \rightarrow \myval{t'}{d}
        }
            
        \infrule[E-StageEta]{
            \myval{t}{d} \rightarrow \myval{t'}{d} 
            \andalso d \le 1
        }{
            \myval{\mynext{\myprev{t}}}{d} \rightarrow \myval{t'}{d}
        }
    \end{frame}

    \begin{frame}{Auxilary rules}
        
    \end{frame}

    \begin{frame}{Future work}
        \begin{itemize}
            \item{TODO}
        \end{itemize}
    \end{frame}

    \begin{frame}{References}
        \printbibliography
    \end{frame}
\end{document}


